\documentclass[letterpaper,11pt]{article}
%------------------------------------------------------------------------------------------------------
%----------------------------------------------------- REQUIRED PACKAGES
\usepackage{latexsym}
\usepackage[empty]{fullpage}
\usepackage{titlesec}
\usepackage{marvosym}
\usepackage[usenames,dvipsnames]{color}
\usepackage{verbatim}
\usepackage{enumitem}
\usepackage[hidelinks]{hyperref}
\usepackage[english]{babel}
\usepackage{tabularx}
\usepackage[skins]{tcolorbox}
\usepackage{graphicx}
\usepackage{etoolbox}
\usepackage{dashrule}
\usepackage{multirow,tabularx}
\usepackage{changepage}
\usepackage{fontawesome}
\usepackage[margin=2cm]{geometry}
\usepackage{lipsum}
\usepackage{microtype}
%\usepackage{flowfram}
\usepackage{multicol}
\usepackage{titlesec}

\RequirePackage{comment}
\definecolor{SwishLineColour}{HTML}{88AC0B}
\definecolor{MarkerColour}{HTML}{006ea0}

\usepackage{tikz}
\usetikzlibrary{arrows}

\usepackage{xcolor}
\colorlet{accent}{NavyBlue}
\colorlet{heading}{black}
\colorlet{emphasis}{black}
\colorlet{body}{black!80!white}

\usepackage{fancyhdr}
\pagestyle{fancy}
\fancyhf{}
\fancyfoot{}
\renewcommand{\headrulewidth}{0pt}
\renewcommand{\footrulewidth}{0pt}

% Adjust margins
\addtolength{\oddsidemargin}{-0.5in}
\addtolength{\evensidemargin}{-0.5in}
\addtolength{\textwidth}{1in}
\addtolength{\topmargin}{-0.3in}
\addtolength{\textheight}{1.0in}

\urlstyle{same}

\raggedbottom
\raggedright
\setlength{\tabcolsep}{0in}

%----------------------------------------------------- Sections formatting
\titleformat{\section}{
  \bfseries\large}{\textcolor{NavyBlue}{\rule[-7pt]{5pt}{1.2em}}}{5pt}{}
  [\color{NavyBlue}\titlerule \vspace{-2pt}]

%----------------------------------------------------- Custom commands
\newcommand{\resumeItem}[2]{
  \item\small{
    \textbf{#1}{: #2 \vspace{-2pt}}
  }
}

\newcommand{\resumeSubheading}[4]{
  \vspace{-1pt}\item
    \begin{tabular*}{0.97\textwidth}[t]{l@{\extracolsep{\fill}}r}
      \textbf{#1} & #2 \\
      \textit{\small#3} & \textit{\small #4} \\
    \end{tabular*}\vspace{-5pt}
}

\newcommand{\resumeSubSubheading}[2]{
    \begin{tabular*}{0.97\textwidth}{l@{\extracolsep{\fill}}r}
      \textit{\small#1} & \textit{\small #2} \\
    \end{tabular*}\vspace{-5pt}
}

\newcommand{\itemmarker}{{\small\textbullet}}
\newcommand{\ratingmarker}{\faCircle}

\newcommand{\resumeSubItem}[2]{\resumeItem{#1}{#2}\vspace{-4pt}}
\renewcommand{\labelitemii}{$\circ$}
\newcommand{\resumeSubHeadingListStart}{\begin{itemize}[leftmargin=*]}
\newcommand{\resumeSubHeadingListEnd}{\end{itemize}}
\newcommand{\resumeItemListStart}{\begin{itemize}}
\newcommand{\resumeItemListEnd}{\end{itemize}\vspace{-5pt}}

\newcommand{\cvtag}[1]{%
  \tikz[baseline]\node[anchor=base,draw=body!40,rounded corners,inner xsep=1ex,inner ysep =0.75ex,text height=1.5ex,text depth=.25ex]{#1};
}

\newcommand{\cvskill}[2]{%
\begin{tabular}{m{8em} c}
\textcolor{emphasis}{\textbf{#1}} &
\foreach \x in {1,...,5}{%
  \space{\ifnumgreater{\x}{#2}{\color{body!30}}{\color{accent}}\faCircle}}\par \\%
\end{tabular}
}

\newcommand{\divider}{\textcolor{body!30}{\hdashrule{\linewidth}{0.6pt}{0.6pt}}\medskip}
\renewenvironment{quote}{\color{accent}\itshape\large}{\par}

\newcommand{\cvsection}[2][]{%
  \bigskip%
  \ifstrequal{#1}{}{}{\marginpar{\vspace*{\dimexpr1pt-\baselineskip}\raggedright\input{#1}}}%
  {\color{heading}\large\bfseries\MakeUppercase{#2}}\\[-1ex]%
  {\color{heading}\rule{\linewidth}{1pt}\par}\medskip
}

\newcommand{\cvsubsection}[1]{%
  \smallskip%
  {\color{emphasis}\large\bfseries{#1}\par}\medskip
}

\newcommand{\cvevent}[4]{%
  {\large\color{emphasis}#1\par}
  \smallskip
  \textbf{\color{accent}#2}\par
  \smallskip
  {\small\makebox[0.5\linewidth][l]{\faCalendar \hspace{0.5em}#3}%
  \ifstrequal{#4}{}{}{\makebox[0.5\linewidth][l]{\faMapMarker\hspace{0.5em}#4}}\par}
  \medskip
}

\newcommand{\makefield}[2]{\makebox[1.5em]{\color{MarkerColour!80!black}#1} #2\hspace{2em}}

\newcommand{\cvref}[3]{%
  \smallskip
  \textcolor{emphasis}{\textbf{#1}}\par
  \begin{description}[font=\color{accent},style=multiline,leftmargin=1.25em]
  \item[\faEnvelope] #2
  \item[\faPhone] #3
  \end{description}
%   \medskip
}


%\title{Curriculum Vitae}

%------------------------------------------------------------------------------------------------------
%----------------------------------------------------- CV STARTS
\begin{document}

%----------------------------------------------------- HEADING
\leftheader{%
  {\LARGE\bfseries\sffamily NIBALDO ALFONSO PINO ARAYA} \\
  \vspace{0.1in}
  {\Large\bfseries\sffamily Data Scientist} \\
  \vspace{0.1in}
  \begin{tabular}{lll}
     \makefield{\faCalendar}{\texttt{04/octubre/1987}} & 
     \makefield{\faMapMarker}{Leonor Cepeda #952, Independencia} & 
     \makefield{\faEnvelope}{\texttt{n.pino@uc.cl}} \\
     \makefield{\faMobile}{\texttt{(+56) 9 75168704}} & 
    \makefield{\faLinkedin}{\href{http://www.linkedin.com/in/nibaldopinoaraya/}{nibaldopinoaraya}} &
    \makefield{\faGithub}{\href{https://github.com/CyberIngeniero}{CyberIngeniero}} \\ %\makefield{\faTwitter}{\texttt{@CyberMath4}} & & \\
  \end{tabular}
}

% primera columna
\begin{minipage}[t]{0.55\textwidth}
\hfill
%----------------------------------------------------- ABOUTME
\section{Acerca de mi}
Soy un profesional apasionado, curioso y ambicioso, dinámico, de mente abierta, y de ágil aprendizaje, con una fascinación particular por los desafíos y el conocimiento. Disfruto del trabajo en equipo y colaborativo. Poseo más de 2 años en experiencia en Analitycs  y más de 6 años de experiencia en el mundo académico.

En particular, poseo experiencia en Machine Learning y Estadísticas (modelos de clasificación, modelos de scoring de riesgo). Así como también poseo conocimientos en diversos lenguajes de programación como R, MATLAB, Python y SQL, entre otros. Mi experiencia se enfoca en desarrollos analíticos para el sector de seguros, donde me desempeño actualmente.


%----------------------------------------------------- EXPERIENCE
\section{Experiencia Laboral}

\cvevent{Actuario de Pricing}{HDI Seguros}{Julio 2019 - Presente}{Santiago de Chile}
\begin{itemize}
\item  \textbf{Modelado:} Diseño e implementación de modelos de riesgo y tarificación para distintos ramos de seguros generales, utilizando diversas herramientas de modelado, predictivo y de clasificación, mediante la implementación de algoritmos de Machine Learning.
\item \textbf{Reportería:} Dentro de mis actividades habituales se encuentra la construcción de reportes para diferentes áreas, utilizando herramientas como Qlikview y Power BI.
\end{itemize}

\divider

\cvevent{Docente de Estadísticas}{Instituto Profesional AIEP}{2018 - 2019}{Santiago de Chile}
\begin{itemize}
\item Docente de asignaturas relacionadas a la estadística (Estadística Aplicada, Inferencia, entro otros) para la Escuela de TIC's. 
\end{itemize}

\divider

\cvevent{Profesor de Matemáticas}{Enseñanza Media y Superior}{2014 - 2019}{Santiago de Chile}
\begin{itemize}
\item Docente de Educación Superior y Educación Media
\end{itemize}

\end{minipage}%
\hfill%
\begin{minipage}[t]{0.4\textwidth}%
% segunda columna
\hfill
%----------------------------------------------------- CURSOS Y CERTIFICACIONES
\section{Cursos y Certificaciones}

\cvevent{Tensorflow in Practice - Specialization}{deeplearning.ai}{agosto 2020}{Coursera}
\divider \\
\cvevent{Scrum Foundation Professional Certificate}{Certiprof}{Julio 2020}{}
\divider \\
\cvevent{SQL for Data Science}{University of California Davis MOOC}{Julio 2020}{Coursera}
\divider \\
\cvevent{Non-Linear Modeling in R with GAMs}{DataCamp}{Julio 2020}{}
\divider \\
\cvevent{Qlikview Developer}{Udemy}{Julio 2019}{}
\divider \\
\cvevent{Curso Introducción a R project}{DataCamp}{Diembre 2017-Febrero 2018}{}

%----------------------------------------------------- SKILLS
\section{Habilidades Informáticas}

\cvskill{R}{5} \\
\cvskill{Python}{5} \\
\cvskill{Qlikview}{5} \\
\cvskill{SQL}{4}
%\cvskill{AWS - GCP}{3}

%----------------------------------------------------- STROGERS
\section{Fortalezas}

\cvtag{Machine Learning} 
\cvtag{Análisis de información}
\cvtag{Modelamiento matemático}
\cvtag{Simulación}
\cvtag{Análisis estadístico}
\cvtag{Visualizaciónes}

\end{minipage}

%----------------------------------------------------- SEGUNDA PAGINA
\begin{minipage}[t]{0.55\textwidth}
% primera columna
\hfill

%----------------------------------------------------- EDUCATION
\section{Formación Académica}

\cvevent{Magíster en Estadísticas}{Pontificia Universidad Católica de Chile}
{Marzo 2018 - Diciembre 2020}{Santiago de Chile}
\divider \\
\cvevent{Analista Data Science con Python}{Desafío Latam}{Julio 2019 - Julio 2020}{Santiago de Chile}
\divider \\
\cvevent{Diplomado en Estadísticas y Métodos Estadísticos}{Pontificia Universidad Católica de Chile}
{Abril 2018 - Diciembre 2018}{Santiago de Chile}
\divider \\
\cvevent{Profesor de Educación Media en Matemáticas e Informática Educativa}{Universidad Católica Cardenal Raúl Silva Henríquez}
{2011 - 2015}{Santiago de Chile}
\divider \\
\cvevent{Ingeniería Civil en Computación e Informática}{Universidad Tecnológica Metropolitana}
{2006 - 2010}{Santiago de Chile}
\divider \\
\cvevent{Bachillerato en Ciencias de la Ingeniería}{Universidad Tecnológica Metropolitana}
{2016 - 2010}{Santiago de Chile}

\end{minipage}%
\hfill%
\begin{minipage}[t]{0.4\textwidth}%
% segunda columna
\hfill
%----------------------------------------------------- IDIOMAS
\section{Idiomas}

\cvskill{Español}{5} \\
\cvskill{Ingles}{3}

%----------------------------------------------------- Referencias
\section{Referencias}
% \cvref{name}{email}{mailing address}

\cvref{Jaime Arroyo Leon}{jaime.arroyo@hdi.cl}{9 3244 5380}
\divider \\
\cvref{Boris Carbonell}{boris.carbonell@gmail.com}{9 6221 5267}
\divider \\
\cvref{Felipe Zapata}{felnepc@gmail.com}{9 6215 8121}
\divider\\
\cvref{Antonio Castillo}{antonio.castillo@udla.cl}{9 8838 8592}
\end{minipage}

\end{document}